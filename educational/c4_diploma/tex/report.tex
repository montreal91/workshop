\documentclass[12pt,a4paper,titlepage]{article}
\usepackage[utf8]{inputenc}
\usepackage[T2A]{fontenc}
\usepackage{amsmath,amssymb,amsopn,amsthm}
\usepackage{graphicx,import}
\usepackage{color}
\usepackage[english,russian]{babel}
\usepackage{array}
\usepackage{titlesec}
\usepackage[titletoc,toc,page]{appendix}
\usepackage{listings}
\usepackage{amsthm}
\usepackage{float}
 \textwidth=17cm
 \textheight=26cm
 \topmargin=-1cm
 \oddsidemargin=0cm
 \headheight=0cm
 \newcommand{\sectionbreak}{\clearpage}
 \renewcommand{\appendixtocname}{Приложение}
 \renewcommand{\appendixpagename}{Приложение}

 \theoremstyle{definition}
 \newtheorem{definition}{Definition}[section]

 % \def\H{\rule{0pt}{1.5ex}H}
 
\lstset{
    breaklines=true,
    basicstyle=\ttfamily,
     postbreak=\raisebox{0ex}[0ex][0ex]{\ensuremath{\color{red}\hookrightarrow\space}}
}
\begin{document}
\begin{titlepage}
\begin{center}
  МИНОБРНАУКИ РОССИИ\\
  Федеральное государственное бюджетное образовательное учреждение\\ высшего образования\\

\bfseries \flqq Челябинский государственный университет\frqq \\
\bfseries (ФГБОУ ВО \flqq ЧелГУ\frqq) \\[0.7cm]

Математический факультет\\
Кафедра вычислительной математики\\[3.4cm]
\large\bfseries Отчёт по преддипломной практике\\[2cm]
% \textit{\large\bfseries{Применение метода иерархий Саати для ранжирования кинофраншиз\\[2cm]}}

{\footnotesize
\begin{tabular}[t]{lllllllllllllllllll}
& & & & & & & & & & & & & & & & &   Выполнил студент       &   Нефедов А.Ю.   \\
& & & & & & & & & & & & & & & & &   академическая группа   &   МП-401        \\
& & & & & & & & & & & & & & & & &   курс                   &   IV            \\
& & & & & & & & & & & & & & & & &   форма обучения         &   очная         \\
& & & & & & & & & & & & & & & & &   направление подготовки &   Прикладная математика    \\
& & & & & & & & & & & & & & & & &                          &    и информатика              \\
& & & & & & & & & & & & & & & & &   подпись                &   \rule{3.3cm}{0.01cm}\\
& & & & & & & & & & & & & & & & &                          &                 \\
& & & & & & & & & & & & & & & & &   дата                   &   "\rule{0,7cm}{0.01cm}"\rule{1cm}{0.01cm} 2018 г.\\
& & & & & & & & & & & & & & & & &                          &                 \\
& & & & & & & & & & & & & & & & &                          &                 \\
& & & & & & & & & & & & & & & & &                          &                 \\
& & & & & & & & & & & & & & & & &                          &                 \\
& & & & & & & & & & & & & & & & &                          &                 \\
& & & & & & & & & & & & & & & & &   Научный руководитель   &   Лепчинский М. Г.\\
& & & & & & & & & & & & & & & & &   Должность              &   доцент     \\
& & & & & & & & & & & & & & & & &                          &   кафедры ВМ  \\
& & & & & & & & & & & & & & & & &   Ученая степень         &   кандидат ф.-м.н.\\
& & & & & & & & & & & & & & & & &   Ученое звание          &                 \\
& & & & & & & & & & & & & & & & &                          &                 \\
& & & & & & & & & & & & & & & & &                          &                 \\
& & & & & & & & & & & & & & & & &   подпись                &  \rule{3.3cm}{0.01cm}\\
& & & & & & & & & & & & & & & & &                          &                 \\
& & & & & & & & & & & & & & & & &   дата                   &  "\rule{0,7cm}{0.01cm}"\rule{1cm}{0.01cm} 2018 г.     \\

\end{tabular}
\\[4.5cm]}

\small{Челябинск}\\
\small{2018}
\end{center}
\end{titlepage}
\tableofcontents
\section{Отчёт}

Общим направлением моей дипломной работы является использование псевдофизической
симуляции для поиска клик в графах.
Идея заключается в том, чтобы представить вершины как материальные точки в
$n$-мерном пространстве, где вершины соединённые ребром притягиваются, а не 
соединённые отталкиваются.

На преддипломную практику была поставлена задача написать программу, реализующую
данную симуляцию и проверить, как работают разные виды взаимодействия вершин.
Цель была достигнута: программа написана и реализовано три вида взаимодействия
вершин в двумерном пространстве:

\begin{itemize}
  \item постоянное: силы притягивания и отталкивания не зависят от расстояния
        между вершинами;
  \item обратно-линейное: силы притягивания и отталкивания обратно
        пропорциональны расстоянию между вершинами; 
  \item "классическое": силы притягивания и отталкивания обратно пропорциональны
        квадрату расстояния между вершинами.
\end{itemize}

Все три вида взаимодействий были протестированы на графах разной сложности.
Здесь под простым графом понимается граф, в котором клики имеют мало общих
вершин.
На достаточно простых случаях все три вида взаимодействия показывают хорошие
результаты (вершины, входящие в клику, слетаются).
Но при постоянном взаимодействии чем больше пересечение клик, тем менее
устойчивы группы слетевшихся вершин.
При обратно-линейном и классическом взаимодействии даже в сложных случаях
слетевшиеся группы вершин достаточно устойчивы и не разлетаются со временем.

\section{Заключение}
Полученные результаты уже свидетельствуют о том, что данный подход можно будет
применять для поиска наибольших клик в графах.
В дальнейшем исследовании планируется ответить на следующие вопросы:

\begin{itemize}
  \item Есть ли зависимость скорости группирования клик от размерности
        пространства?
  \item Как влияет на эффективность работы алгоритма начальное положение вершин?
  \item Насколько эффективен данный подход на больших (100-300 вершин) графах?
\end{itemize}

Плюс необходимо добавить в программу возможность редактирования проверяемого
графа.
\clearpage
\begin{appendices}
\section{Текст программы}

\subsection{vertex.h}
  \lstinputlisting[language=c++]{../src/vertex.h}
\subsection{vertex.cc}
  \lstinputlisting[language=c++]{../src/vertex.cc}
\subsection{application.h}
  \lstinputlisting[language=c++]{../src/application.h}
\subsection{application.cc}
  \lstinputlisting[language=c++]{../src/application.cc}
\subsection{main.cc}
  \lstinputlisting[language=c++]{../src/main.cc}
\end{appendices}

\end{document}
